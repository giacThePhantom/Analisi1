\chapter{Continuit\`a}
$f: X\subset\mathbb{R}\rightarrow\mathbb{R}, x_0\in X$ Se $x_0$ \`e un punto isolato di X f \`e continua nel punto isolato ed esso \`e un punto di accumulazione. Se $x_0$ \`e un punto di accumulazione per X f si dice continua in $x_0$ se $\exists \lim\limits_{x\rightarrow x_0} f(x)\wedge\lim\limits_{x\rightarrow x_0} f(x)=f(x_0)$
\section{Note}
f si dice continua in $x_0\in X \Leftrightarrow\forall\epsilon>0\exists\delta>0 :\forall x\in X:|x-x_0|<\delta \Rightarrow |f(x)-f(x_0)|<\epsilon$\\
Non ha senso parlare di continuit\`a per un punto che non appartiene al dominio della funzione.\\
Dalle proposizioni dei limiti se f e g sono continue in $x_0\in domf\cap domg$ anche somma, prodotto e rapporto sono continue in $x_0$.\\
Applicando il teorema dell'esistenza del limite sinistro e destro per una funzione monotona, si ottiene la continuit\`a nel loro dominio di tutte le funzioni elementari.\\
$\forall x_0\in\mathbb{R},\lim\limits_{x\rightarrow x_0} x^n=x_0^n$
