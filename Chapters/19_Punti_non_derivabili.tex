\chapter{Punti di non derivabilit\`a}
Sia $f:I\rightarrow\mathbb{R}$ una funzione con $I$ intervallo e $x_0\in I$, $f$ sia continua in $x_0$. \\
Si supponga che $\exists f'_+(x_0),f'_-(x_0)\wedge f'_+(x_0)\neq f'_-(x_0)$
\begin{itemize}
\item $(x_0;f(x_0))$ \`e un punto angoloso se almeno una delle due derivate \`e finita:
\begin{itemize}
\item $f'_+(x_0)\in\mathbb{R},f'_-(x_0)\in\mathbb{R}\wedge f'_+(x_0)\neq f'_-(x_0)$
\item $f'_+(x_0)\in\mathbb{R}, f'_-(x_0)=\pm\infty$
\item $f'_+(x_0)=\pm\infty, f'_-(x_0)\in\mathbb{R}$
\end{itemize}
\item $(x_0;f(x_0))$ si dice cuspide se: $f'_+(x_0)=\pm\infty, f'_-(x_0)=\mp\infty$ 
\item $(x_0;f(x_0))$ si dice punto a tangente verticale se $f'_+(x_0)=\pm\infty=f'_-(x_0)=\pm\infty$
\end{itemize}