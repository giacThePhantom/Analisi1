\chapter{Asintoti}
Studio pi\`u dettagliato della crescita di funzioni all'infinito del dominio, ai punti di accumulazione non appartenenti al dominio o ai punti di discontinuit\`a.
\section{Asintoto orizzontale}
Sia $f$ una funzione definita in un intorno di $\pm\infty$, se $\lim\limits_{x\rightarrow\pm\infty}f(x)=b\in\mathbb{R}$ allora la retta $y=b$ si dice asintoto orizzontale per $f
$ per $x\rightarrow\pm\infty$
\section{Asintoto obliquo}
Sia $f$ una funzione definita in un intorno di $\pm\infty$, se esistono $a\neq 0$ e $b\in\mathbb{R}$ tali che$\lim\limits_{x\rightarrow\pm\infty}[f(x)-(ax+b)]=0$ allora la 
retta $y=ax+b$ si dice asintoto obliquo per $f$, per $x\rightarrow\pm\infty$.
\begin{equation}
\lim\limits_{x\rightarrow\pm\infty}[f(x)-(ax+b)]=0\Leftrightarrow
\begin{cases}
\lim\limits_{x\rightarrow\pm\infty}\frac{f(x)}{x}=a\\
\lim\limits_{x\rightarrow\pm\infty}(f(x)-ax)=b
\end{cases}
\end{equation}
\section{Asintoto verticale}
Sia $f:\mathbb{X}\subset\mathbb{R}\rightarrow\mathbb{R}$ una funzione e $x_0$ punto di accumulazione per $domf$. Se $\lim\limits_{x\rightarrow x_0^\pm}f(x)=\pm\infty$ allora
la retta verticale $x=x_0$ si dice asintoto verticale per $f$.
\subsection{Asintoto verticale da destra}
\begin{equation}
\lim\limits_{x\rightarrow x_0^+}f(x)=\pm\infty
\end{equation}
\subsection{Asintoto verticale da sinistra}
\begin{equation}
\lim\limits_{x\rightarrow x_0^-}f(x)=\pm\infty
\end{equation}