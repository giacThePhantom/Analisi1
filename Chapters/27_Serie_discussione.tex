\chapter{Alcune serie}
\section{La serie geometrica}
\begin{equation}
\sum\limits_{k=0}^{+\infty}q^k
\end{equation}
Si ricordi che per $q\neq 1$, $S_n=\sum\limits_{k=0}^n q^k=\frac{1-q^{n+1}}{1-q}$, pertanto:\\
$\lim\limits_{n\rightarrow+\infty}S_n=\lim\limits_{n\rightarrow+\infty}\frac{1-q^{n+1}}{1-q}=\sum\limits_{k=0}^{+\infty}q^k=
\begin{cases}
\frac{1}{1-q}\;\;\;\;|q|<1\\
+\infty\;\;\;\;q\ge 1\\
\nexists\;\;\;\;q\le 1
\end{cases}
$
\section{Serie di Mengoli}
\begin{equation}
\sum\limits_{n=1}^{+\infty}\dfrac{1}{n(n+1)}
\end{equation}
Il termine ennesimo tende a zero, la serie converge ad $1$ ed \`e una serie telescopica.
\subsection{Serie telescopica}
Si definisce una serie telescopica una serie i cui elementi $a_n=b_n-b_{n+1}$, con $\{b_n\}_n$ un'altra successione. Nel caso di Mengoli $b_n=\frac{1}{n}$, pertanto $S_n=b_1-b_{n+1}$.
\section{Serie logaritmica}
\begin{equation}
\sum\limits_{n=1}^\infty \dfrac{1}{n^\alpha(\log n)^\beta}
\end{equation}
Risulta convergente se $\alpha>1,\beta\in\mathbb{R}$ o se $\alpha=1, \beta>1$
\section{Serie armonica}
\begin{equation}
\sum\limits_{n=1}^{+\infty}\dfrac{1}{n}
\end{equation}
Soddisfa $\lim\limits_{n\rightarrow+\infty}a_n=\lim\limits_{n\rightarrow+\infty}\frac{1}{n}=0$, si provi che la coda della serie non \`e infinitesima. Basta osservare che $\forall n_0\in\mathbb{N},\; n_0\ge 1$:\\
$\sum\limits_{n=n_0}^{+\infty}\frac{1}{n}>\sum\limits_{n=n_0+1}^{2n_0}\frac{1}{n}=\frac{1}{n_0+1}+\frac{1}{n_0+2}+\cdots+\frac{1}{2n_0}\ge \frac{1}{2n_0}n_0\ge\frac{1}{2}$,\\
Pertanto $\Rightarrow\lim\limits_{n_0\rightarrow+\infty}\sum\limits_{n=n_0}^{+\infty}\frac{1}{n}\neq 0$. Pertanto dal teorema segue che la serie non \`e convergente, ed essendo che la 
serie \`e a termini positivi si pu\`o concludere che $\sum\limits_{n=1}^{+\infty}\frac{1}{n}=+\infty$.
\subsection{Serie armonica generalizzata}
\begin{equation}
\sum\limits_{n=1}^{+\infty}\dfrac{1}{n^\alpha}
\end{equation}
\subsubsection{Se $\alpha\le 1$}
In questo caso la serie \`e divergente:\\
\begin{enumerate}
\item Se $\alpha=1$ si \`e gi\`a dimostrato che la serie armonica \`e divergente,
\item Se $0<\alpha<1\Rightarrow 0<\frac{1}{n}<(\frac{1}{n})^\alpha\;\;\forall n>1$. Da $\sum\limits_n\frac{1}{n}=+\infty$, per il teorema del confronto $\sum\limits_n\frac{1}{n^
\alpha}=+\infty$
\item Se $\alpha\le 0$, allora $\sum\limits_{n=1}^{+\infty}\frac{1}{n^\alpha}=\sum\limits_{n=1}^{+\infty}n^{-\alpha}=+\infty$ in quanto $\lim\limits_{n\rightarrow+\infty}n^{-
\alpha}=1$ se $\alpha=0$ o $=+\infty$ se $\alpha<0$
\end{enumerate}
\subsubsection{Se $1<\alpha<2$}
$\sum\limits_{n=1}^{+\infty}\frac{1}{n^\alpha}$, $f(x)=\frac{1}{x^\alpha}:[1;+\infty[\rightarrow]0;+\infty[$, $\alpha>0$, $f(x)$ decrescente $\int_1^{+\infty}\frac{1}{x^\alpha}dx<+\infty\Leftrightarrow\alpha>1$.
\subsubsection{Se $\alpha=2$}
Basta osservare che $0<\frac{1}{n^2}\le\frac{2}{n(n+1)}\;\;\forall n\ge 1$ ($\Leftrightarrow n(n+1)\le 2n^2\Leftrightarrow n\le n^2$). Si conosca che $\sum\limits_n \frac{1}{n(n+1)}<+\infty$ (serie di Mengoli), dal criterio del confronto segue che $\sum\limits_{n=1}^{+\infty}\frac{1}{n^2}<+\infty$.
\subsubsection{Se $\alpha\ge 2$}
In questo caso la serie \`e convergente: basta osservare che: $0<\frac{1}{n^\alpha}\le\frac{1}{n^2}$, poich\`e $\sum\limits_{n=1}^{+\infty}\frac{1}{n^2}<+\infty$, per il criterio del 
confronto si ha $\sum\limits_{n=1}^{+\infty}\frac{1}{n^\alpha}<+\infty$ se $\alpha\ge 2$.
\subsection{Scrittura della serie armonica generalizzata}
\begin{equation}
\sum\limits_{n=1}^{+\infty}\frac{1}{n^\alpha}=
\begin{cases}
+\infty\;\;\;\;\alpha\le 1\\
<+\infty\;\;\;\;\alpha> 1\\
\end{cases}
\end{equation}
\section{Serie di potenze}
Sia $\{a_n\}_n$ una successione di numeri reali con $x_0\in\mathbb{R}$ fissato. $\forall x\in\mathbb{R}$, la serie:
\begin{equation}
\sum\limits_{n=0}^{+\infty}a_n(x-x_0)^n
\end{equation}
Si dice serie di potenze. Si dice insieme di congruenza $E=\{x\in\mathbb{R}:$ la serie di potenze risulta convergente$\}$.
\subsection{Note}
\begin{itemize}
\item $E\neq\emptyset$: $x_0\in E$ sempre.
\item Esistono serie per cui $E=\{x_0\}$.
\end{itemize}
\subsection{Raggio di convergenza}
Si dice raggio di convergenza della serie $\sum\limits_{n=0}^{+\infty}a_n(x-x_0)^n$ il valore $r=\sup\limits_{x\in E}|x-x_0|$, $r\in[0;+\infty[$, $r=+\infty$ dove $E$ \`e 
l'insieme di convergenza della serie.
\subsection{Teorema}
Sia $r$ il raggio della serie $\sum\limits_n a_n(x-x_0)^n$, allora:
\begin{itemize}
\item la serie converge assolutamente $\forall x\in\mathbb{R}:|x-x_0|<r$.
\item la serie non converge $\forall x\in\mathbb{R}:|x-x_0|>r$.
\item Nei punti $x=x_0+r$ e $x=x_0-r$ il suo carattere non \`e determinato.
\end{itemize}
\subsection{Determinazione del raggio di convergenza}
Sia $\{a_n\}_n$ la successione di numeri reali e $\sum\limits_n a_n(x-x_0)^n$ la serie di potenze. Se esiste
\begin{itemize}
\item $\lim\limits_{n\rightarrow+\infty}\sqrt[n]{|a_n|}=l\;\;\;\;0\le l\le+\infty$ oppure
\item $\lim\limits_{n\rightarrow+\infty}|\frac{a_{n+1}}{a_n}|=l\;\;\;\;0\le l\le+\infty$
\end{itemize}
La serie di potenze ha raggio di convergenza:\\
$r=\begin{cases}
+\infty\;\;\;\; l=0\\
0\;\;\;\;l=+\infty\\
\frac{1}{l}\;\;\;\;l\in\mathbb{R}
\end{cases}
$
\section{Serie di Taylor}
Se $f\in C^\infty(]a;b[)$ (derivabile all'infinito con continuit\`a), $x_0\in]a;b[$. Si pu\`o scrivere:
\begin{equation}
\sum\limits_{k=0}^\infty \dfrac{f^{(k)}(x_0)}{k!}(x-x_0)^k
\end{equation}
\subsection{Definizione}
$f\in C^\infty(]a;b[)$ si dice sviluppabile in serie di Taylor di centro $x_0$ in $I$ se $\exists$ un intorno $I$ di $x_0$ tale che:
\begin{itemize}
\item[i] la serie di Taylor di $f$ centrato in $x_0$ \`e convergente $\forall x\in I$.
\item[ii] la sua somma \`e $f(x)$.
\subsection{Condizioni di sviluppabilit\`a}
$f$ \`e sviluppabile in serie di Taylor $\Rightarrow$
\begin{equation}
E_n(x)=f(x)-P_n(x)\rightarrow 0
\end{equation}
Ovvero si deve stimare il resto di Lagrange $E_n(x)=\frac{f^{(n+1)}(c_x)}{(n+1)!}(x-x_0)^{n+1}$ e determinare se vale $0$. Stimando il resto di Lagrange si dimostrano facilmente
che per $x_0=0$ valgono gli sviluppi notevoli dimostrati precedentemente.
\end{itemize}